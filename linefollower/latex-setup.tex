% Configuração do cabeçalho e rodapé
\usepackage{fancyhdr}
\pagestyle{fancy}
\renewcommand{\headrulewidth}{1pt}
\fancyhead[L]{\leftmark}
\fancyhead[C]{}
\fancyhead[R]{\thepage}
\fancyfoot{}

% Configuração do idioma e hifenização
\usepackage[brazilian, provide=*]{babel}
\usepackage[brazilian=nohyphenation]{hyphsubst}
\addto\captionsbrazilian{\renewcommand{\contentsname}{Sumário}}

% Pacotes para formatação e elementos visuais
\usepackage{graphicx}
\usepackage{epigraph}
\usepackage{xcolor}
\usepackage{listings}
\usepackage{float}
\usepackage{mathtools}
\usepackage{amsmath}
\usepackage{tikz}
\usepackage{amssymb}
\usepackage{array}
\usepackage{qtree}
\usepackage{forest}
\usepackage{booktabs}
\usepackage{tcolorbox}

% Definição de cores personalizadas
\definecolor{contentbg}{rgb}{0.9, 0.9, 0.9}
\definecolor{titlebg}{rgb}{0.2, 0.4, 0.8}

% Configuração das caixas de texto
\tcbuselibrary{theorems}
\tcbset{
    myboxstyle/.style={
        colback=contentbg,
        colframe=black,
        coltitle=white,
        colbacktitle=titlebg,
        fonttitle=\bfseries,
        boxrule=0.5mm,
        width=\linewidth,
        arc=4mm
    }
}

% Configuração do estilo de código
\lstset{
    language=Haskell,
    basicstyle=\ttfamily\small,
    keywordstyle=\color{blue},
    commentstyle=\color{gray},
    stringstyle=\color{red},
    numbers=left,
    numberstyle=\tiny\color{gray},
    stepnumber=1,
    numbersep=5pt,
    frame=single,
    showspaces=false,
    showstringspaces=false,
    breaklines=true,
    breakatwhitespace=true,
    tabsize=2
}

% Ajustes de layout
\renewcommand{\floatpagefraction}{.8}
\renewcommand{\textfraction}{.1}